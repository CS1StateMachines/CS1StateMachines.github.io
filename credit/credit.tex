\documentclass[11pt]{article}
\usepackage{calc,fancyhdr,lastpage}
\usepackage[hmargin=.75in,vmargin=1in,
            footskip=.55in,headsep=.55in-\headheight]{geometry}
\usepackage{amsmath,amssymb}

% Formats, symbols, abbreviations.
\let\altemph\textsl
\let\strong\textbf
\let\code\texttt
\let\latinabb\emph
\newcommand*{\etc}{\latinabb{etc}}
\newcommand*{\eg}{\latinabb{e.g.}}
\newcommand*{\ie}{\latinabb{i.e.}}
% To get proper-looking symbols in \texttt.
\newcommand*{\txtbksl} {\symbol{"5C}}% \
\newcommand*{\txtcaret}{\symbol{"5E}}% ^
\newcommand*{\txtunder}{\symbol{"5F}}% _
\newcommand*{\txtlcurl}{\symbol{"7B}}% {
\newcommand*{\txtrcurl}{\symbol{"7D}}% }
\newcommand*{\txttilde}{\symbol{"7E}}% ~

% Commands \question[marks]{title} and \subquestion[marks]{title}.
\newcounter{questionnumber}
\newcommand*{\question}[2][]
   {\refstepcounter{questionnumber}\section*
    {Part \thequestionnumber.\quad
        \ifx\empty#1\empty\else[#1 marks]\quad\fi#2}}
\newcounter{subquestionnumber}[questionnumber]
\renewcommand*{\thesubquestionnumber}{\alph{subquestionnumber}}
\newcommand*{\subquestion}[2][]
   {\refstepcounter{subquestionnumber}\subsubsection*
    {Subpart (\thesubquestionnumber)\quad
       \ifx\empty#1\empty\else[#1 marks]\quad\fi#2}}

% Redefine `enumerate' to use less vertical space.
\let\etaremune\enumerate
\let\etaremunedne\endenumerate
\renewenvironment{enumerate}
   {\etaremune
    \setlength{\topsep}{.25ex plus .125ex minus .1825ex}%
    \setlength{\itemsep}{\topsep}\setlength{\parsep}{0ex}%
    \setlength{\leftmargin}{1.75em}\setlength{\labelsep}{.5em}%
    \setlength{\labelwidth}{1.75em}\ignorespaces}
   {\etaremunedne}

% Redefine `itemize' to use less vertical space.
\let\ezimeti\itemize
\let\ezimetidne\enditemize
\renewenvironment{itemize}
   {\ezimeti
    \setlength{\topsep}{.25ex plus .125ex minus .1825ex}%
    \setlength{\itemsep}{\topsep}\setlength{\parsep}{0ex}%
    \setlength{\leftmargin}{1.75em}\setlength{\labelsep}{.5em}%
    \setlength{\labelwidth}{1.75em}\ignorespaces}
   {\ezimetidne}

%% A heading in the instructions.
\newcommand*{\heading}[1]{\subsubsection*{#1}}

% Headings.
\pagestyle{fancy}
\let\headrule\empty
\let\footrule\empty
\lhead{{\bfseries CS1}}
\chead{{\bfseries\large Credit Card Simulator}}
\rhead{{\bfseries Fall 2xxx}}
\lfoot{{Anonymous}}
\cfoot{{}}
\rfoot{{Page \thepage\ of \pageref{LastPage}}}


\begin{document}

\vspace*{-4.5ex}
\noindent
\strong{Due:  }
\hfill
\strong{Worth: }



\heading{Submitting your project}

You must hand in your work electronically, using Gradescope.
Log in to
\begin{verbatim}
    https://www.gradescope.com/anonymous
\end{verbatim}

For this project, you must hand in just one file:
\begin{itemize}
\item \code{credit.py}
\end{itemize}

{}  You can submit a new version of the file at any time (though the 
lateness penalty applies if you submit after the deadline)---look in the 
``Replace'' column.  For the purposes of determining the lateness penalty, 
the submission time is considered to be the time of your latest submission.


\heading{The \code{if \txtunder\txtunder name \txtunder\txtunder == \txtunder\txtunder"main"\txtunder\txtunder} block}
All your testing code should be inside the \code{if \txtunder\txtunder name \txtunder\txtunder == \txtunder\txtunder"main"\txtunder\txtunder} block. All you global variables must be initialized outside of the \code{if \txtunder\txtunder name \txtunder\txtunder == \txtunder\txtunder"main"\txtunder\txtunder} block.


\heading{Hints \& tips}

\begin{itemize}
\item
    Start early.  Programming projects always take more time than you 
    estimate!
\item
    Do not wait until the last minute to submit your code.  You can 
    overwrite previous submissions with more recent ones, so submit early 
    and often---a good rule of thumb is to submit every time you get one 
    more feature implemented and tested.
\item
    Write your code incrementally.  Don't try to write everything at once, 
    and then compile it.  That strategy never works.  Start off with 
    something small that compiles, and then add functions to it gradually, 
    making sure that it compiles every step of the way.
\item
    Read these instructions and make sure you understand them thoroughly 
    before you start---ask questions if anything is unclear!
\item
    Inspect your code before submitting it.  Also, make sure that you 
    submit the correct file.
\item
    Seek help when you get stuck!  Check the discussion board first to see 
    if your question has already been asked and answered.  Ask your 
    question on the discussion board if it hasn't been asked already.  Talk 
    to your TA during the lab if you are having difficulties with 
    programming.  Go to the instructors' office hours if you need extra 
    help with understanding the course content.

    At the same time, beware not to post anything that might give away any 
    part of your solution---this would constitute plagiarism, and the 
    consequences would be unpleasant for everyone involved!  If you cannot 
    think of a way to ask your question without giving away part of your 
    solution, then please drop by office hours or ask by email instead.
\item
    If your email to the TA or the instructor is ``Here is my program.  
    What's wrong with it?'', don't expect an answer!  We expect you to at 
    least make an effort to start to debug your own code, a skill which you 
    are meant to learn as part of this course.  And as you will discover 
    for yourself, reading through someone else's code is a difficult 
    process---we just don't have the time to read through and understand 
    even a fraction of everyone's code in detail.

    However, if you show us the work that you've done to narrow down the 
    problem to a specific section of the code, why you think it doesn't 
    work, and what you've tried to fix it, it will be much easier to 
    provide you with the specific help you require and we will be happy to 
    do so.
\end{itemize}

\heading{Testing}

Below is a guide to how to write your testing code.

You should include code that tests the functions that you have written to 
make sure that they match the project specifications.  Make sure that you 
test your functions thoroughly.  That means that you should make sure that 
your functions work for all the different possible scenarios.  Mindlessly 
plugging in various parameter values is not enough---it's not the quantity 
of tests that matters, it's having tests that cover all of the possible 
scenarios, and that requires thinking about possible scenarios.

{}  The documentation of the testing strategy should include, for each 
function you test and for each test case, a description of what output your 
function should produce, and a brief explanation of why the test case and 
output are significant in verifying the correctness of the program.

\heading{Documentation}



When writing code, you must write documentation to describe what your code 
is doing.  Documentation helps others and yourself understand what your 
code is meant to do.  The general rule of thumb for documentation states 
that you should add comments to your code in the following situations:
\begin{itemize}
\item
    For every function, as a docstring, to describe the parameters of the 
    function and what the function does.  See below for more details on 
    docstrings.
\item
    Before every global variable declaration, to describe what kind of 
    information the variable stores and what properties (if any) that 
    information is supposed to have throughout the execution of the code.
\item
    Before all complicated sections of code, to help the reader understand 
    what that code section is trying to do.
\item
    In general, comments should \emph{not} simply restate what the code 
    does (this does not add any useful information to the code).  Comments 
    should \emph{add} information that is implicit in the code, \eg, about 
    what purpose a computation serves, or why a certain section of code is 
    written the way it is.
\end{itemize}

\heading{Style}

Good style practices should be adhered to when writing your code.  This 
includes the following:
\begin{itemize}
\item
    Use Python style conventions for your function and variable names.  In 
    particular, please use ``pothole case'': lowercase letters with words 
    separated by underscores (\code{\txtunder}), to improve readability.
\item
    Choose good names for your functions and variables.  For example, 
    \code{num\txtunder coffee\txtunder cups} is more helpful and readable 
    than \code{ncc}.
\item
    To make sure your program will be formatted correctly is never 
    to mix spaces and tabs when indenting ---use only tabs, or only spaces.
\item
    Put a blank space before and after every operator.  For example:
\begin{verbatim}
    b = 3 > x and 4 - 5 < 32  # good style: easy to read
    
    b= 3>x and 4-5<32         # bad style: hard to read
\end{verbatim}
\item
    Write a docstring comment for each function.  (See below for guidelines 
    on the content of your docstrings.) 
    Put a blank line after every docstring comment.
\item
    Each line must be less than 80 characters long, including tabs and 
    spaces.  You should break up long lines using \code{\txtbksl}. 
\item
    Your code should be readable and readily understandable.
\end{itemize}

\heading{Guidelines for writing docstrings}

\begin{itemize}
\item  Describe precisely \emph{what} the function does.
\item  Do not reveal \emph{how} the function does it.
\item  Make the purpose of every parameter clear.
\item  Refer to every parameter by name.
\item  Be clear about whether the function returns a value, and if so, 
    what.
\item  Explain any conditions that the function assumes are true.  
    Examples: ``\code{n is an int}'', ``\verb|n != 0|'', ``\code{the height 
    and width of p are both even}''.
\item  Be concise.
\item  Ensure that the text you write is grammatically correct.
\item  Write the docstring as a command (\eg, ``Return the first \ldots'') 
    rather than a statement (\eg, ``Returns the first \ldots'').
\end{itemize}


\newpage
\chead{{\bfseries\large Project \#\,1
    --- Credit Card Simulator}}

\noindent

For this project, you will implement a simulator for credit card transactions. You will maintain the  balance owed on the credit card by keeping track of  new purchases, the interest accrued, and bill payments that can sometimes be partial. In addition, you will implement a simple flagging algorithm, and make sure that the card is deactivated (i.e., no further purchases can be made) if fraud is suspected.

The simulation proceeds as a series of operations, which are simulated using calls to the functions that you will define. For example, a simulation might proceed as follows:

\begin{verbatim}
if __name__ == '__main__':
    initialize()
    purchase(80, 8, 1, "Canada")
    print("Now owing:", amount_owed(8, 1))
    pay_bill(50, 2, 2)
    print("Now owing:", amount_owed(2, 2))
    print("Now owing:", amount_owed(6, 3))
    purchase(40, 6, 3, "Canada")
    print("Now owing:", amount_owed(6, 3))
    pay_bill(30, 7, 3)
    print("Now owing:", amount_owed(7, 3))
    print("Now owing:", amount_owed(1, 5))
\end{verbatim}

The lines in the code above correspond to purchases, bill payments, and inquiries about amounts owed. The passage of time is simulated by having dates in the function calls.

The credit card account operates according to the following rules.
\begin{itemize}
	\item Initially, the amount owed is 0.
	\item The amount owed is divided into two parts: the amount that is accruing interest, and the amount that is not accruing interest. The only money that is not accruing interest during the month is the money spent on purchases during that same month. Any other money owed is accruing interest.
	\item An interest of 5\% is added to the amount owed in the last second of each month. Assume that no purchases are made between the time the interest is added to the amount owed and the time that the month changes.
	\item When the credit card bill is paid, and the amount owed is not paid in full, the payment first goes to pay the amount that is accruing interest, and only then to pay the amount that is not accruing interest.
	\item If the card is used for purchases in three different countries in a row, the card is deactivated. The third purchase does not work, and no further purchases can be made. 

For example, if two purchases are made in Canada and the United States, and the next attempted purchase is in France, the third purchase does not go through, and no money is added to the amount owed. If another attempt to make a purchase is then made in Canada, again, the purchase attempt fails and no money is added to the amount owed.

On the other hand, if purchases are made in Canada, then France, then Canada, then Canada, then the United States, and then the United States, all the purchases go through, since at no point were purchases made in three different countries in a  row.
\end{itemize}

\newpage

\noindent Here is a  sample output of a simulation, along with comments explaining the output.
\begin{verbatim}
if __name__ == '__main__':
    initialize()
    purchase(80, 8, 1, "Canada")
    print("Now owing:", amount_owed(8, 1))      #80.0
    pay_bill(50, 2, 2)
    print("Now owing:", amount_owed(2, 2))      #30.0     (=80-50)
    print("Now owing:", amount_owed(6, 3))      #31.5     (=30*1.05)
    purchase(40, 6, 3, "Canada")
    print("Now owing:", amount_owed(6, 3))      #71.5     (=31.5+40)
    pay_bill(30, 7, 3)
    print("Now owing:", amount_owed(7, 3))      #41.5     (=71.5-30)
    print("Now owing:", amount_owed(1, 5))      #43.65375 (=1.5*1.05*1.05+40*1.05)
    purchase(40, 2, 5, "France")
    print("Now owing:", amount_owed(2, 5))      #83.65375 
    print(purchase(50, 3, 5, "United States"))  #error    (3 diff. countries in 
                                                #          a row)
                                                
    print("Now owing:", amount_owed(3, 5))      #83.65375 (no change, purchase
                                                #          declined)
    print(purchase(150, 3, 5, "Canada"))        #error    (card disabled)
    print("Now owing:", amount_owed(1, 6))      #85.8364375 
                                                #(43.65375*1.05+40)
                                            

\end{verbatim}


{}  We provide you with a ``starter'' version of \code{credit.py}---a 
skeleton of the code you will have to write, with some parts already 
filled in.  Please read it carefully and make sure you understand 
everything in the starter code \altemph{before} you start making changes! 
You must not change the function signatures. 

{}  When designing and testing your functions, you may assume that every date given is a valid date in 2020 (there is no need to check that). However, you may not assume that dates used later in the simulation always occur chronologically after dates that were used earlier.


\question{}

Implement the following functions in \code{credit.py}.  Note that the 
names of the functions are case-sensitive and must not be changed.  You are 
not allowed to change the number of input parameters.  Doing so will cause 
your code to fail when run with our testing programs, so that you will not 
get any marks for functionality.

Note that ``iff" means ``if and only if." When we say that a function \verb;f; returns True iff condition \verb;c; holds, we mean that \verb;f; returns \verb;True; if \verb;c; holds, and \verb;False; if it doesn't hold.

\subquestion
	{\code{date\txtunder same\txtunder or\txtunder later(day1, month1, day2, month2)}}

This function returns \verb;True; iff the date \verb;(day1, month1); is the same as the date \verb;(day2, month2);, or occurs later than \verb;(day2, month2);.  Assume the dates given are valid dates in the year 2020.

\subquestion
	{\code{all\txtunder three\txtunder different(c1, c2, c3)}}

This function returns \verb;True; iff the values of the three strings \verb;c1;, \verb;c2;, and \verb;c3; are all different from each other.

\subquestion
	{\code{purchase(amount, day, month, country)}}
This function simulates a purchase of amount \verb;amount;, on the date \verb;(day, month);, in the country \verb;country; (given as a capitalized string). The function should return the string \verb;"error"; and not have any other effect (except for possibly disabling the card) if any of the following conditions obtain:

\begin{itemize} 
	\item There already was a simulation operation on a date later than \verb;(day, month); (e.g., a purchase or a check for the amount owed).
	\item The card is becoming disabled due to the current attempted purchase, or is already disabled.
\end{itemize}

You may assume that \verb;amount; is greater than 0 and that \verb;country; is a valid country name.

\subquestion
	{\code{amount\txtunder owed(day, month)}}
This function returns the amount owed as of the date \verb;(day, month);

This function returns the string \verb;"error";  if there already was a simulation operation on a date later than \verb;(day, month); (e.g., a purchase or a check for the amount owed).


\subquestion
	{\code{pay\txtunder bill(amount, day, month)}}
This function simulates the payment of the amount owed on the credit card. When the credit card bill is paid, and the amount owed is not paid in full, the payment first goes to pay the amount that is accruing interest, and only then to pay the amount that is not accruing interest. 

You may assume that \verb;amount; is greater than 0.

This function returns the string \verb;"error";  if there already was a simulation operation on a date later than \verb;(day, month); (e.g., a purchase or a check for the amount owed).



\question{}

In the \verb;if  __name__ == "__main__"; block, add code that tests your functions. While you may run large simulations as well, your job is to come up with a testing strategy that ensures that your code works according to the project specifications. This is best done by testing every individual aspect of the behaviour of the code. Add comments to clarify the testing strategy: the goal is to make sure that it is possible to look at the testing code that you wrote and the comments that you have added, and be convinced that you have tested for all the categories of the typical cases, and all the categories of the boundary/edge cases. 

\end{document}
