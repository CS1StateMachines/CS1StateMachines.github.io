\documentclass[11pt]{article}
\usepackage{calc,fancyhdr,lastpage}
\usepackage{listings}
\usepackage[hmargin=.75in,vmargin=1in,
            footskip=.55in,headsep=.55in-\headheight]{geometry}
\usepackage{amsmath,amssymb}
\usepackage{hyperref}
\usepackage{graphicx}


% Formats, symbols, abbreviations.
\let\altemph\textsl
\let\strong\textbf
\let\code\texttt
\let\latinabb\emph
\newcommand*{\etc}{\latinabb{etc}}
\newcommand*{\eg}{\latinabb{e.g.}}
\newcommand*{\ie}{\latinabb{i.e.}}
% To get proper-looking symbols in \texttt.
\newcommand*{\txtbksl} {\symbol{"5C}}% \
\newcommand*{\txtcaret}{\symbol{"5E}}% ^
\newcommand*{\txtunder}{\symbol{"5F}}% _
\newcommand*{\txtlcurl}{\symbol{"7B}}% {
\newcommand*{\txtrcurl}{\symbol{"7D}}% }
\newcommand*{\txttilde}{\symbol{"7E}}% ~

% Commands \question[marks]{title} and \subquestion[marks]{title}.
\newcounter{questionnumber}
\newcommand*{\question}[2][]
   {\refstepcounter{questionnumber}\section*
    {Question \thequestionnumber.\quad
        \ifx\empty#1\empty\else[#1 marks]\quad\fi#2}}
\newcounter{subquestionnumber}[questionnumber]
\renewcommand*{\thesubquestionnumber}{\alph{subquestionnumber}}
\newcommand*{\subquestion}[2][]
   {\refstepcounter{subquestionnumber}\subsubsection*
    {Part (\thesubquestionnumber)\quad
       \ifx\empty#1\empty\else[#1 marks]\quad\fi#2}}

% Redefine `enumerate' to use less vertical space.
\let\etaremune\enumerate
\let\etaremunedne\endenumerate
\renewenvironment{enumerate}
   {\etaremune
    \setlength{\topsep}{.25ex plus .125ex minus .1825ex}%
    \setlength{\itemsep}{\topsep}\setlength{\parsep}{0ex}%
    \setlength{\leftmargin}{1.75em}\setlength{\labelsep}{.5em}%
    \setlength{\labelwidth}{1.75em}\ignorespaces}
   {\etaremunedne}

% Redefine `itemize' to use less vertical space.
\let\ezimeti\itemize
\let\ezimetidne\enditemize
\renewenvironment{itemize}
   {\ezimeti
    \setlength{\topsep}{.25ex plus .125ex minus .1825ex}%
    \setlength{\itemsep}{\topsep}\setlength{\parsep}{0ex}%
    \setlength{\leftmargin}{1.75em}\setlength{\labelsep}{.5em}%
    \setlength{\labelwidth}{1.75em}\ignorespaces}
   {\ezimetidne}

%% A heading in the instructions.
\newcommand*{\heading}[1]{\subsubsection*{#1}}

% Headings.
\pagestyle{fancy}
\let\headrule\empty
\let\footrule\empty
\lhead{{\bfseries CS1}}
\chead{{\bfseries\large Lab \#2}}
\rhead{{\bfseries Fall 2xxx}}
\lfoot{{Anonymous}}
\cfoot{{}}
\rfoot{{Page \thepage\ of \pageref{LastPage}}}


\begin{document}
\lstset{language=Python}

For this lab, you will write a simple pocket calculator program. The program will be able
to display the current value on the screen of the calculator. You will store the current
value of the calculator in a variable. The initial value of the calculator (i.e., the initial
current value) is 0.
Do not write out the entire lab assignment and only then try to debug it: this almost never
works. If you're new to programming, you shouldn't, as a rule, write more than five lines
of code between trying the new code out to see if it does something sensible.
You will get credit for the lab if you make reasonable progress toward completing it.
Credit may be given for programs that accomplish only some of the tasks assigned.
The TAs are here to help you. If you are stuck, ask for help!

\question{Warmup}

\textit{If you are confident about functions, global and local variables, and the distinction between print and return, you may skip this question and proceed to Question 2.}

Start with the following code in Pyzo:

\begin{verbatim}
def my_sqrt(x):
    sqr = x**.5
    return sqr	
	
if __name__ == "__main__":
    res = my_sqrt(25)
	
\end{verbatim}

\subquestion{}
When you run this code, nothing is output to the screen. Explain why. Now, add a line of code inside the \texttt{main} block so that the result of the computation in \texttt{my\_sqrt} is printed.

\subquestion{}
If you add the line \texttt{print(sqr)} inside the \texttt{main} block, running the program produces an error. Explain why.

How can you modify the function \texttt{my\_sqrt} so that \texttt{print(sqr)} doesn't produce an error?

\subquestion{}
Write a function with the signature \texttt{my\_print\_square(x)} which \textbf{prints} (rather than returns) the square of the argument \texttt{x}. Call this function from the \texttt{main} block, and explain the difference between the effect of calling \texttt{my\_print\_square(x)} and calling \texttt{my\_sqrt(x)}.

What is the output for the following code, if it is run from within the main block? Explain.

\begin{verbatim}
	res = my_print_square(25)
	print(res)
\end{verbatim}

\question{Welcome Message}

In the \verb;if __name__ == "__main__";  block, write code that displays the following:
\begin{verbatim}
Welcome to the calculator program.
Current value: 0
\end{verbatim}





\question{Displaying the Current Value}
Write a function whose signature is
\verb;display_current_value();,
and which displays the current value in the calculator. In the  \verb;if __name__ == "__main__";  block, test this function by
calling it and observing the output.

\question{Addition}

Write a function whose signature is \verb;add(to_add);, and which adds \verb;to_add; to the current value in the calculator, and modifies the current value accordingly. In the  \verb;if __name__ == "__main__’";  block, test the function \verb;add; by calling it, as well as by calling \verb;display_current_value();. \textbf{Hint}: when modifying global variables
from within functions, remember to declare them as \verb;global;.


\question{Multiplication}
Write a function whose signature is \verb;mult(to_mult);, and which multiplies  the current value in the calculator  by \verb;to_mult;, and modifies the current value accordingly. In the  \verb;if __name__ == "__main__";  block, test the function.

\question{Division}
Write a function whose signature is \verb;div(to_div);, and which divides  the current value in the calculator  by \verb;to_div;, and modifies the current value accordingly. In the  \verb;if __name__ == "__main__";  block, test the function.  What values
of \verb;to_div; might cause problems? Try them to see what happens.


\question{Memory and Recall}
Pocket calculators usually have a memory and a recall button. The memory button
saves the current value and the recall button restores the saved value. Implement
this functionality.

\question{Undo}
Implement a function that simulates the Undo button: the function restores the
previous value that appeared on the screen before the current one.

Pressing the Undo twice restores the original value:

\begin{verbatim}
    # current value: 25
    add(5)	# current value: 30
    mult(2) # current value: 60
    undo()  # current value: 30
    undo()  # current value: 60
    undo()  # current value: 30
\end{verbatim}




\end{document}
